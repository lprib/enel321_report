\documentclass[12pt]{article}

\usepackage{amsmath}
\usepackage{anysize}
\usepackage{amssymb}
\usepackage{pgf}
\usepackage{gensymb}
\usepackage{hyphenat}
\usepackage{geometry}

\usepackage{caption}
\usepackage{subcaption}
\usepackage{float}

\usepackage[T1]{fontenc}

\title{ENEL321 Lab Report}
\author{Liam Pribis 81326653}
\date{May 4th, 2020}
\marginsize{4.5em}{4.5em}{3em}{3em}

\begin{document}
\clearpage\maketitle
\thispagestyle{empty}
\newpage
\setcounter{page}{1}
\section{Summary}
During this lab, 5 different controllers were designed and simulated for use in controlling the roll angle or a model rocket. The controllers were first modeled to determine their probable response, and then run on a test rig. The response of each controller was measured and their performance was analysed. It was found that PID controllers performed better than PD or P controllers.

\section{Methodology}
Five controllers were used in this lab: one strictly proportional (P) controller, one proportional-derivative (PD), and three proportional-integral-derivative (PID) controllers.
Specific gains were generally selected by fixing the other gains of the system and allowing the gain in question to vary over a range. The best controller in this range was selected.
This was done using Python, SciPy, and MatPlotLib.
\subsection{Proportional Controller}
Multiple values for the proportional gain in this controller were tested. Lower values of $K_p$
resulted in less overshoot and rise time, as illustrated in Figure \ref{fig:pcontroller}, but also had a large negative steady state error. Large values of $K_p$ achieved small steady state errors, but the overshoot quickly became too large as $K_p$ increased. The large overshoot and rise time is due to the fact that when $K_p$ is large and the error is large (at $t = 10$), the commanded angle of the canards will also be very large ($u_p(t) = K_p(r - x)$). The large canard angle will cause the rocket to gain rotational velocity very quickly (resulting in small $t_r$). Once the rocket reaches the commanded 90\degree  rotation, it's rotational inertia means it will overshoot before beginning to turn back towards the commanded angle. This process may repeat if $K_p$ is large, resulting in oscillation. The negative correlation between $K_p$ and $e_{ss}$ is due to the fact that there is an internal spring in the model of the rocket system. The $K_p$ term is always trying to push the rocket's rotation toward the commanded angle. Larger $K_p$ allows the canards to output more force (due to a larger angle of attack) to counteract this spring, and reduce $e_{ss}$.

Because of these constraints, a $K_p$ of 0.3 was chosen, which produced a measured $M_p$ of $25\%$ and an $e_{ss}$ of -13\degree.
\begin{figure}[H]
\centering
\begin{minipage}{.5\textwidth}
    \centering
    \captionsetup{justification=centering, margin=0.5cm}
    \scalebox{0.55}{\input{pcontroller_gain.pgf}}
    \caption{Modeled P control with varying $K_p$ and step input.}
    \label{fig:pcontroller}
\end{minipage}%
\begin{minipage}{.5\textwidth}
    \centering
    \captionsetup{justification=centering, margin=0.5cm}
    \scalebox{0.55}{\input{pdcontroller_gain.pgf}}
    \captionof{figure}{Modeled PD control with $K_p$ = 0.8, varying $K_d$, and step input.}
    \label{fig:pdcontroller}
\end{minipage}
\end{figure}

\subsection{Proportional-Derivative Controller}
When designing a PD controller, a $K_p$ of 0.8 was chosen due to the tradeoff of $M_p\%$ and $e_{ss}$ as previously discussed. Because a derivative term is present, the $K_p$ can be higher than
in the P controller without compromising on overshoot. The derivative term dampens the oscillations present in the proportional-only controller, decreasing the overshoot for
equivalent values of $K_p$. The dampening effect is due to the fact that the $K_d$ term always tries to counteract change in the commanded angle ($u_d(t) = K_d(0-\dot{x})$). As with $K_p$, the $K_d$ term also presents a tradeoff. Larger $K_d$ values will add more damping, which decrease the overshoot or remove it entirely
(resulting in an overdamped system). Smaller $K_d$ values decrease damping, causing a faster rise time $t_r$, but more overshoot. This is illustrated in Figure \ref{fig:pdcontroller}.
A value of 0.3 was chosen to compromise on rise time and overshoot.



\subsection{Proportional-Integral-Derivative Controllers}
Three PID controllers were chosen. Controller 2 was chosen with the maximum $K_p$ value allowed in the specifications, to decrease $t_r$ as much as possible.
In a model, $K_d$ was then increased until it suitably removed the overshoot, and no more. Having a larger $K_d$ than necessary could cause an overdamped system.
Overdamping is undesirable because past the point of critical damping, there is no overshoot at all, and any additional $K_d$ will only cause a decrease in $t_r$.

Controller 3 was chosen with a $K_p$ of 1.0, which is outside of the range provided in the specifications. Controller 3 also had a smaller $K_d$ than controller 2.
Both of these decisions were made to minimise rise time, at the expense of other parameters.

Controller 5 was chosen with a $K_p$ of 0.5 and a $K_d$ of 0.12. Both of these values are smaller than the $K_p$ and $K_d$ values of the other PID controllers. In modelling, controller 5 had a shorter rise time, but also a large overshoot, which is due to its low $K_d$ value.

For all three PID controllers, a $K_i$ value of 0.1 was chosen. Through simulation, it was found that $K_i = 0.1$ reduced steady state error to almost zero. Larger values of $K_i$ increase the overshoot because from $t=10$ to $t=t_p$, the difference between the commanded and the measured angle is large. This positive difference is summed in each time delta by the controller ($u_i(t) = K_i\int_{0}^{t}r-x d\tau$, and $r-x$ is positive), adding to the positive angular momentum of the rocket. When the rocket reaches its commanded set point, the increased momentum means a larger overshoot, demonstrated in Figure \ref{fig:pidcontroller}.

\begin{figure}[h]
    \centering
    \scalebox{0.55}{\input{pidcontroller_ki.pgf}}
    \captionof{figure}{Modeled PID control with $K_p = 0.8$, $K_d = 0.6$, varying $K_i$, and step input.}
    \label{fig:pidcontroller}
\end{figure}

\section {Results}

\begin{table}[H]
    \centering
    \caption{Measured parameters for the controllers.\label{tab:params}}
    \begin{tabular}{||c||c|c c c|c c c c c c c c||}
        \hline
        Controller & Type & $K_p$ & $K_i$ & $K_d$ & $t_r$ & $t_p$ & $t_s$ & $M_p\%$ & $e_{ss}$(\degree) & $\xi$ & $\omega_d$ & $\omega_n$\\
        \hline
        1 & P &0.3 & 0.0 & 0.00 & 0.80 & 11.90 & 19.88 & 25.15 & -12.96 & 0.40 & 1.70 & 1.85 \\
        2 & PID & 0.8 & 0.1 & 0.60 & 1.34 & 14.52 & 19.72 & 4.00 & 1.28 & 0.72 &*&*\\
        3 & PID & 1.0 & 0.1 & 0.40 & 0.70 & 11.42 & 12.86 & 3.69 & -0.24 & 0.72 &*&*\\
        4 & PD & 0.8 & 0.0 & 0.30 & 0.62 & 11.40 & 19.88 & 3.01 & -6.81 & 0.74 &*&*\\
        5 & PID & 0.5 & 0.1 & 0.12 & 0.64 & 11.68 & 16.60 & 24.32 & -0.44 & 0.41 & 1.90 & 2.08 \\
        \hline
        specs & & $\leq 0.8$ & & & < 0.9 & & & < 5 & < 1\degree & > 0.4 & &\\
        \hline
    \end{tabular}
    \newline
    {* Oscillations were not clear enough to determine $\omega_d$ or $\omega_n$.}
\end{table}

\subsection{Controller 1: $K_p = 0.3$}
Controller 1 used only  proportional control. As expected, it produced a large overshoot and steady state error (from Table \ref{tab:params}), due to lack of damping ($K_d$) and integral control ($K_i$). The integral term would reduce the steady state error because as time goes on, if there is a recurring positive or negative error, it will be summed by the integral, causing the $K_i$ term to have a larger effect over time. The derivate control would reduce the overshoot and oscillation, because it counteracts and change (derivative) in the response.

\begin{figure}[h]
\centering
\begin{minipage}{.5\textwidth}
    \centering
    \captionsetup{justification=centering, margin=0.5cm}
    \scalebox{0.55}{\input{response1.pgf}}
    \caption{Controller 1 response.}
    \label{fig:response1}
\end{minipage}%
\begin{minipage}{.5\textwidth}
    \centering
    \captionsetup{justification=centering, margin=0.5cm}
    \scalebox{0.55}{\input{response2.pgf}}
    \captionof{figure}{Controller 2 response.}
    \label{fig:response2}
\end{minipage}
\end{figure}

\subsection{Controller 2: $K_p = 0.8, K_i = 0.1, K_d = 0.6$}
Controller 2 produced a low overshoot/$M_p\%$, a low steady state error, and a high rise time (from Table \ref{tab:params}). The rise time and steady state error did not reach specifications. The high rise time, high steady state error, and low overshoot are due to the high $K_d$ gain, causing damping in the response. As shown in Figure \ref{fig:response2}, controller 2 shows almost no oscillation and a reduced gradient as it approaches 90\degree. The damping also causes this controller to fail to reach a low enough steady state error, because the $K_d$ term is resisting the change toward a 90\degree rotation. To optimise this controller, the derivative gain could potentially be reduced some, depending on how much overshoot can be tolerated.


\subsection{Controller 3: $K_p = 1.0, K_i = 0.1, K_d = 0.4$}
Controller 3 met all specifications except $K_p \leq 0.8$. It produced a fast rise time, very little overshoot, and almost zero steady state error. The main problem with this control scheme is that it saturates the angle of the rocket's canards. This is shown in Figure \ref{fig:commanded3}. At $t=10$, the controller attempts to actuate the fins to >90\degree. This is in contrast to controller 5 (Figure \ref{fig:commanded5}), which never actuates the canards more than 50\degree.

While this response looks good in the model, in the real world the canards would stall and send the rocket in to an unpredictable state. Saturating the control surfaces also makes this controller very difficult to model accurately, because it becomes non-linear.

\begin{figure}[h]
    \centering
    \begin{minipage}{.5\textwidth}
        \centering
        \captionsetup{justification=centering, margin=0.5cm}
        \scalebox{0.55}{\input{commanded3.pgf}}
        \caption{Controller 3 canard actuation.}
        \label{fig:commanded3}
    \end{minipage}%
    \begin{minipage}{.5\textwidth}
        \centering
        \captionsetup{justification=centering, margin=0.5cm}
        \scalebox{0.55}{\input{commanded5.pgf}}
        \captionof{figure}{Controller 5 canard actuation.}
        \label{fig:commanded5}
    \end{minipage}
    \end{figure}
\begin{figure}[h]
\centering
\begin{minipage}{.5\textwidth}
    \centering
    \captionsetup{justification=centering, margin=0.5cm}
    \scalebox{0.55}{\input{response3.pgf}}
    \caption{Controller 3 response.}
    \label{fig:response3}
\end{minipage}%
\begin{minipage}{.5\textwidth}
    \centering
    \captionsetup{justification=centering, margin=0.5cm}
    \scalebox{0.55}{\input{response4.pgf}}
    \captionof{figure}{Controller 4 response.}
    \label{fig:response4}
\end{minipage}
\end{figure}

\subsection{Controller 4: $K_p = 0.8, K_d = 0.3$}
    Controller 4 meets all specifications except that of the steady state error. The large steady state error is due to the lack of an integral term. Since there is no integral to accumulate or sum the recurring error, it stays constant at -6.8\degree.

\begin{figure}[h]
    \centering
    \scalebox{0.55}{\input{response5.pgf}}
    \captionof{figure}{Controller 5 response.}
\end{figure}

\section{Conclusion}
The controllers that best met the specifications on the test rig were PID controllers. The strictly proportional controller had a large overshoot and oscillation. The P and PD controllers both had issues with steady state error. One of the PID controllers (controller 2) had a derivative gain that was too large. The excess of damping caused by the derivative gain resulted in a long rise time and relatively large steady state error. If a second iteration of this controller were to be designed, the derivative gain would be reduced.

One of the PID controllers had acceptable parameters, but its large $K_p$ caused it to saturate the actuators. The saturation was not modelled for, causing the response of the system to look normal. If this controller were tested in a physical wind tunnel, it likely would not behave in the same way, showing the importance of physical testing for many control systems.

\end{document}